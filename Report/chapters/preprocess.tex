
\فصل{‌پیش‌پردازش داده‌ها}
\پرش‌بلند
\قسمت{اصلاح ساختار پروتئین}
با توجه به نقص‌‌هایی که ممکن است در روش اندازه گیری به کار گرفته شده وجود داشته باشد، نیاز است که یک سری اصلاحات اولیه‌ای روی ساختار پروتئین انجام شود و فایل \پسف مورد نیاز تهیه شود. بدین منظور میتوان از ماژول‌‌های نرم افزار \ومدی نیز استفاده کرد که برای این کار در نوار ابزار به بخش \لر{Extensions} رفته و در بخش \لر{Modeling} گزینه \لر{Automatic PSF Builder} را انتخاب کرد.
\figs{autopsf}{ماژول نرم افزار \ومدی}{}{0.25}
در این جا روش دوم که با استفاده از کد نویسی در‌ترمینال است شرح داده می‌شود.
در ابتدا اتم‌‌های پروتئین انتخاب شده و در متغیر \verb*|ubq| ریخته میشوند. سپس فایل \پدب مدنظر ساخته می‌شود.
\begin{latin}
	\verb|set ubq [atomselect top protein]|
	
	\verb|$ubq writepdb ubqp.pdb|
\end{latin}
سپس پکیج‌‌های مورد نیاز بارگزاری میشوند و فایل توپولوژي و تغییر نام‌‌های مورد نیاز برای نرم افزر \ومدی که بتواند قابل استفاده باشد انجام میگیرد.
\begin{latin}
	\verb|package require psfgen|
	
	\verb|topology top_all27_prot_lipid.inp|
	
	\verb|pdbalias residue HIS HSE|
	
	\verb|pdbalias atom ILE CD1 CD|
	
	\verb|segment U {pdb ubqp.pdb}|
\end{latin}
در قدم بعدی مختصات اتم‌‌های هیدروژن حدس زده شده و سپس فایل‌‌های \پدب و \پسف ساخته میشوند.
\begin{latin}
	\verb|guesscoord|
	
	\verb|writepdb ubq.pdb|
	
	\verb|writepsf ubq.psf|
\end{latin}
\قسمت{قرار دادن پروتئین در \لر{Water Box}}
حال لازم است که پروتئین را در یک جعبه آب قرار بدهیم تا در ادامه به ریلکس کردن آن بپردازیم. برای این منظور میتوان از ماژول‌‌های نرم افزار \ومدی و یا کد نویسی استفاده کرد. به منظور استفاده از ماژول‌ها کافیست که در نوار ابزار به بخش \لر{Extensions} رفته و در بخش \لر{Modeling} گزینه \لر{Add Solvation Box} را انتخاب کرد. و در صفحه باز شده فایل \پدب و \پسف مد نظر را انتخاب کرد.
\figs{solvate}{استفاده از ماژول نرم افزار \ومدی}{}{0.25}
در این تمرین از روش دوم یعنی نوشتن کد‌ها استفاده می‌شود. برای اینکار از بخش \لر{Extensions} گزنه \لر{TK console} انتخاب می‌شود که‌ترمینال برنامه‌نویسی در نرم افزار \ومدی است. سپس دستورات زیر وارد می‌شود.
\begin{latin}
\verb|package require solvate|
\end{latin}
ابتدا پکیج مورد نیاز فراخوانی می‌شود.

\begin{latin}
\verb|solvate ubq.psf ubq.pdb -t 5 -o ubq_wb|\\
\end{latin}
در این دستور، گزینه \verb*|-t| فاصله‌ای که باید پروتئین در بزرگترین بعد خود از دیواره باکس خود بگیرد را تعیین میکند. و گزینه \verb*|-o| نشانگر این است که فایل‌‌های خروجی باید پیشوند مشخص شده را داشته باشند.\\
\begin{latin}
\verb|set everyone [atomselect top all]|
\end{latin}
\begin{latin}
	\verb|measure minmax $everyone|
\end{latin}
\begin{latin}
	\verb|min : {10.442000389099121 7.947000026702881 -5.327000141143799}|
	
	\verb|max : {51.11899948120117 50.23099899291992 40.895999908447266}|\\
\end{latin}
در این دو دستور ابتدا همه اتم‌ها انتخاب و سپس دورترین و نزدیکترین نقاط نسبت به مبدا محاسبه میشوند. که میتوان با استفاده از این داده‌ها مختصات وسط پروتئین را نیز محاسبه کرد.
\begin{equation}
	[x_c, y_c, z_c] = \frac{min + max}{2}
\end{equation}
\begin{equation}
	[x_c, y_c, z_c] = [30.775, 29.085, 17.785]
\end{equation}
میتوان برای اینکار از دستور زیر هم استفاده کرد.
\begin{latin}
	\verb|measure center $everyone|
\end{latin}
سپس اینبار اتم‌‌های پروتئین انتخاب شده و در یک فایل \پدب ذخیره میشوند.
\begin{latin}
	\verb|set selprotein [atomselect top protein]|
	
	\verb|$selprotein writepdb common/ubq_ww_eq.pdb|
\end{latin}
حال میتوان این فایل را در نرم افزار \ومدی باز کرد.
\figs{box}{پروتئین احاطه شده در یک مکعب پر از آب}{arg3}{0.25}
\فصل{پردازش پروتئین}
\پرش‌بلند
\قسمت{ریلکس کردن پروتئین با استفاده از \نمدی}
حال که پروتئن در ظرف آب قرارگرفت لازم است که با استفاده از نرم افزار \نمدی آن را ریلکس کرد. بدین منظور در پوشه‌ای که دارای فایل‌‌های کانفیگوریشن مربوطه به \نمدی است رفته و بررسی میکنیم که آدرس فایل‌ها درست تنظیم شده باشند. که در اینجا بایستی پوشه \verb*|common| باشد. پس لازم است که فایل‌‌های مربوط به جعبه آب و فایل‌‌های \پدب و \پسف در پوشه مربوطه کپی شود.
\figs{copy}{قرار دادن فایل‌‌های مورد نیاز در پوشه مربوطه}{arg3}{0.25}

حال به پوشه دارای فایل تنظیمات رفته و‌ترمینال را در آن اجرا میکنیم،
\figs{conf}{فایل تنظیمات}{}{0.25}
 سپس با اجرای دستور زیر عملیات ریلکسیشن روی پروتئن انجام می‌شود.
\begin{latin}
	\verb|namd2 ubq_wb_eq.conf > log.txt|
\end{latin}
این دستور، نرم افزار \نمدی را اجرا کرده و سپس نتیجه خروجی را در علاوه بر اینکه در‌ترمینال نشان می‌دهد، در فایل لاگ نیز ذخیره میکند.
\figs{log}{نتیجه خروجی ریلکسیشن}{arg3}{0.25}
نتایج در پوشه \verb|1-3-box| ذخیره میشوند.
\قسمت{اعمال نیروی ثابت به پروتئین}
حال که پروتئین ریلکس شده است میتوان محاسبات مد نظر را روی آن انجام داد. در اینجا ابتدا اتم کربن از \رزیجوی اول را ثابت کرده و سپس درجهت برداری که اتم کربن \رزیجوی اول را به اتم کربن \رزیجوی آخر متصل میکند، نیروی ثابت به پروتئین اعمال می‌شود. بدین منظور لازم است که بردار اعمال نیرو محاسبه شود. این بردار از تفریق مختصات اتم آخر از اتم اول بدست می آید.

ابتدا اما لازم است که پارامتر‌‌های مربوط به ساکن بودن یا متحرک بودن \رزیجوی را با پارامتر \verb*|beta| مشخص کرد. اگر این پارامتر دارای مقدار ۱ باشد به معنی ثابت بودن است و مقدار ۰ به معنی امکان تحرک داشتن.
\begin{latin}
	\verb|set allatoms [atomselect top all]|
	
	\verb|$allatoms set beta 0|
	
	\verb|set fixedatom [atomselect top "resid 1 and name CA"]|
	
	\verb|$fixedatom set beta 1|

	\verb|$allatoms set occupancy 0|
	
	\verb|set smdatom [atomselect top "resid 76 and name CA"]|
	
	\verb|$smdatom set occupancy 11.54|
\end{latin}
 نیرو‌‌های بین اتمی نیز با پارامتر \verb*|occupancy| تعیین می‌شود که در اینجا تمامی اتم‌ها به غیر از اتم کربن آخرین \رزیجوی نیروی بین اتمی شان مقدار ۰ قرار داده شده است.
 
 حال برای محاسبه جهت بردار نیرو ابتدا مختصات اتم آخر را از اتم اول کم کرده و سپس آن را تقسیم بر اندازه اش کرده تا بردار نرمال بدست آید. این کار با استفاده از این دستور ممکن است و میتوان آن را نیز به صورت دستی محاسبه کرد.
 ابتدا به صورت دستی محاسبه میکنیم و سپس نتیجه را با استفاده از دستور مقایسه میکنیم.
مختصات اتم کربن \رزیجوی اول و آخر.
\begin{latin}
	\verb|resid 1 CA : {26.240999 24.927999 3.279000}|
	\verb|resid 76 CA : {36.798000 39.930000 32.453998}|
\end{latin}
حال این دو مختصات شروع و پایان را از همدیگر کم میکنیم و بر اندازه این بردار تقسیم میکنیم.
\begin{equation}
	v = [-10.5557001 -15.00200 -29.174998]
\end{equation}
\begin{equation}
	|v| = 27.157748
\end{equation}
\begin{equation}
	\frac{v}{|v|} = 0.388, 0.5542, 1.074
\end{equation}
برای محاسبه از دستور‌‌های نرم افزار \ومدی نیز میتوان استفاده کرد.
\begin{latin}
	\verb|set smdpos [lindex [$smdatom get {x y z}] 0]|
	
	\verb|set fixedpos [lindex [$fixedatom get {x y z}] 0]|
	
	\verb|vecnorm [vecsub $smdpos $fixedpos]|
\end{latin}
در این دستور ابتدا پارامتر‌‌های مربوط به مختصات اتم ثابت و متحرک در یک متغیر ذخیره شده و سپس در دستور آخر ابتدا این دو پارامتر را از هم دیگر کم کرده و سپس نرمال آن را حساب میکند.
\begin{latin}
	\verb|{0.3063295295636324 0.43530883445678653 0.8465627194491839}|
\end{latin}
حال این پارامتر‌‌های بدست آمده در متغیر‌‌های جدا گانه $x, y, z$ ذخیره میشوند.
\begin{latin}
	\verb|set x 0.30632|
	
	\verb|set y 0.43530|
	
	\verb|set z 0.84656|
\end{latin}
در نمایش پروتئین به این صورت تغییری‌ایجاد می‌شود.
\figs{xyz}{افزودن نقاط جدید به ساختار پروتئین}{arg3}{0.25}
\قسمت{ایجاد تغییرات در فایل \lr{conf}}
بدین منظور ساختار پوشه‌ها را بدین صورت تغییر میدهیم که فایل‌‌های ساختاری و مختصاتی در پوشه \verb*|common| و فایل‌‌های \verb*|.conf| را در پوشه \verb*|pull_cf| قرار میدهیم.
\figs{common}{ساختار فایل‌ها در پوشه \لر{common}}{arg3}{0.25}
در همین صفحه نتایج حاصل از اجرای نرم افزار \نمدی هم مشاهده می‌شود.
\figs{pullcf}{ساختار فایل‌ها در پوشه \لر{pullcf}}{arg3}{0.25}
به منظور انجام پردازش‌‌های لازم توسط نرم افزار \نمدی لازم است که دستورات و شرایط اولیه لازم در فایل \verb*|.conf| داده شود.

در بخش اول \verb|Adjustable Parameters| مسیر مربوط به فایل‌‌های ساختاری و مختصات و پیشوند فایل‌‌های ذخیره را مشخص میکنیم. سپس دمایی که در آن واکنش قرار است انجام بگیرد را مشخص میکنیم. سپس در پارامتر \verb|firsttimestep| مشخص میکنیم که از کدام استپ زمانی شروع به بررسی کند.
\figs{adjust}{تنظیمات اجرای شبیه‌سازی‌ \نمدی}{arg3}{0.25}
سپس در مرحله بعد سایر تنظیمات شامل مشخص کردن فایل توپولوژی و غیره را با 0 یا 1 کردن پارامتر داخل \verb*|if| تنظیم میکنیم.
\figs{simul}{پارامتر‌‌های شبیه‌سازی‌}{arg3}{0.25}
سپس در نهایت با مشخص کردن فایل نهایی که دارای بردار نیرو است و اینکه ما میخواهیم به چه مقداری این شبیه‌سازی‌ انجام شود، فایل را ذخیره میکنیم.
\figs{timestep}{تعیین پارامتر‌‌های مربوط به شبیه‌سازی‌}{arg3}{0.25}
سپس در نهایت به منظور اجرای شبیه‌سازی‌ در لینوکس در پوشه‌ای که حاوی فایل \verb*|.conf| بود دستور زیر را وارد میکنیم که به معنای اجرای شبیه‌سازی‌ با استفاده از پارامتر‌‌های موجود در فایل تنظیمات و ذخیره سازی لاگ آن است.
\begin{latin}
	\verb|namd2 sample.conf > log.txt|
\end{latin}
نتایج شبیه‌سازی‌ را میتوان از طریق‌ترمینال در حین اجرای نرم افزار نیز مشاهده کرد.
\figs{log}{لاگ نمایش داده شده در صفحه‌ترمینال}{arg3}{0.25}
در نهایت میتوان نتایج حرکت پروتئین را با بارگذاری فایل \verb*|.dcd| بر روی یک فایل ساختاری بدست آورد.
\figs{frame}{41 فریم از حرکت پروتئین}{arg3}{0.25}
