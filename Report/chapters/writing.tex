


\فصل{پیش پردازش داده ها}
\پرش‌بلند
\قسمت{تعیین \لر{Water Box}}
حال لازم است که پروتئین را در یک جعبه آب قرار بدهیم تا در ادامه به ریلکس کردن آن بپردازیم. برای این منظور میتوان از ماژول های نرم افزار \ومدی و یا کد نویسی استفاده کرد. به منظور استفاده از ماژول ها کافیست که در نوار ابزار به بخش \لر{Extensions} رفته و در بخش \لر{Modeling} گزینه \لر{Add Solvation Box} را انتخاب کرد. و در صفحه باز شده فایل \پدب و \پسف مد نظر را انتخاب کرد.
\figs{solvationbox}{استفاده از ماژول نرم افزار \ومدی}{}{0.3}
در این تمرین از روش دوم یعنی نوشتن کد ها استفاده میشود. برای اینکار از بخش \لر{Extensions} گزنه \لر{TK console} انتخاب میشود که ترمینال برنامه نویسی در نرم افزار \ومدی است. سپس دستورات زیر وارد میشود.\\
\verb|package require solvate|\\
ابتدا پکیج مورد نیاز فراخوانی میشود.\\
\verb|solvate ubq.psf ubq.pdb -t 5 -o ubq_wb|\\
در این دستور، گزینه \verb*|-t| فاصله ای که باید پروتئین در بزرگترین بعد خود از دیواره باکس خود بگیرد را تعیین میکند. و گزینه \verb*|-o| نشانگر این است که فایل های خروجی باید پیشوند مشخص شده را داشته باشند.\\
\verb|set everyone [atomselect top all]|

{10.442000389099121 7.947000026702881 -5.327000141143799} {51.11899948120117 50.23099899291992 40.895999908447266}



>Main< (tutorials) 43 % set smdpos [lindex [$smdatom get {x y z}] 0]
36.79800033569336 39.93000030517578 32.45399856567383
>Main< (tutorials) 44 % set smdpos [lindex [$fixedatom get {x y z}] 0]
26.240999221801758 24.92799949645996 3.2790000438690186
>Main< (tutorials) 45 % set smdpos [lindex [$smdatom get {x y z}] 0]
36.79800033569336 39.93000030517578 32.45399856567383
>Main< (tutorials) 46 % set fixedpos [lindex [$fixedatom get {x y z}] 0]
26.240999221801758 24.92799949645996 3.2790000438690186
>Main< (tutorials) 47 % vecnorm [vecsub $smdpos $fixedpos]
0.3063295295636324 0.43530883445678653 0.8465627194491839


پس از اینکه نرم افزار \ومدی را در‌ترمینال لینوکس با وارد کردن دستور \verb*|vmd| در پوشه مدنظر اجرا شد، با استفاده از منوی فایل، فایل مربوط به \verb*|ubq.pdb| که در دایرکتوری
\begin{flushleft}
\lr{tutorials/1-1-build/example-output/ubq.pdb}
\end{flushleft}
 وجود دارد بارگذاری می‌شود.
\\
\figs{2}{وارد کردن مولکول جدید}{}{0.25}
در مرحله بعد به منظور اعمال نیرو روی مولکول بر روی مولکول کلیک راست کرده و گزینه \لر{Load Data into Molecule} را انتخاب کرده و سپس فایل مربوط به اعمال نیروی کششی ثابت به پروتئین یوبیکوئیتین را از این دایرکتوری انتخاب میکنیم.
\begin{flushleft}
	\lr{tutorials/3-2-pullcf/example-output/ubq\_ww\_pcf.dcd}
\end{flushleft}
\figs{4}{وارد کردن داده‌‌های مربوط به نیرو}{}{0.25}
با انجام این کار \اع{41} فریم روی مولکول لود می‌شود.
در مرحله بعد به منظور محاسبه میزان کرنش بایستی که فاصله بین دو \رزیجوی ابتدا و انتها را در طی این زمان ثبت کرد. به همین منظور در منوی \لر{Graphics} و در گزینه \لر{Representations} با کلیک بر روی دکمه \لر{Create Rep} دو حالت نمایش برای این مولکول‌ایجاد میکنیم. یکی برای \رزیجوی اول و دیگری برای آخرین \رزیجوی، حالت نمایش این دو را نیز در تب \لر{Drawing Styles} به حالت \لر{CPK} تغییر میدهیم.
\\
\figs{6}{ایجاد کردن دو حالت نمایش}{}{0.25}
\figs{5}{تغییر حالت نمایش به }{}{0.25}
سپس با استفاده از دو جدول در انتهای پنجره مربوط به \لر{Keywords} و \لر{Value} دو \رزیجوی اول و آخر انتخاب میشوند. در سیستم عامل اوبونتو شماره \رزیجوی ‌ها از 0 شروع می‌شود.
\figs{7}{انتخاب شماره \رزیجوی برای نمایش دادن}{}{0.3}
سپس حالت رنگ آمیزی در تب \لر{Draw Style} و با گزینه \لر{Coloring Method} به حالت \لر{Element} تغییر داده می‌شود تا اتم‌‌های نیتروژن در ابتدا و اتم کربن در انتها به عنوان‌ترمینال‌‌های شروع و پایان انتخاب شوند.
\figs{8}{تغییر حالت رنگ آمیزی بر اساس نوع عنصر}{arg3}{0.25}
در رنگ آمیزی بر حسب نوع عنصر، اتم نیتروژن به رنگ آبی پررنگ و اتم کربن به رنگ سبز نمایش داده می‌شود. برای \رزیجوی اول اتم نیتروژن و برای \رزیجوی آخر اتم کربن انتخاب می‌شود.

حال برای اندازه گیری فاصله بین این دو اتم از روی تب \لر{Mouse} گزینه \لر{Label} و سپس \لر{‌‌‌Bond} انتخاب می‌شود. سپس دو اتم مد نظر را انتخاب کرده و در نهایت در تب \لر{Graphics} گزینه \لر{Labels} و در لیست بالا گزینه \لر{Bonds} انتخاب می‌شود.
\figs{9}{محاسبه فاصله پیوند بین دو اتم}{}{0.25}
سپس با انتخاب پیوند و با زدن تیک \لر{Show Preview} گراف را بدست آورده با کلیک کردن روی گزینه \لر{Save} داده‌‌های بدست آمده در محل دلخواه و با فرمت دلخواه ذخیره میشوند.
\figs{10}{بدست آوردن گراف فاصله}{}{0.25}
\figs{scatter}{اندازه پیوند}{}{0.5}
برای فیت کردن یک چند جمله‌ای درجه 8 بر روی این دیتا در پایتون از دستور زیر استفاده شد.
\begin{latin}
	\begin{lstlisting}[language=Python]
		plt.figure(figsize = (10,10))
		xp = np.linspace(0, 41)
		p = np.poly1d(np.polyfit(x, y, 8))
		plt.plot(x, y, '.', xp, p(xp), '-')
	\end{lstlisting}
\end{latin}
نتیجه بدست آمده در تصویر زیر آمده است.
\figs{poly}{فیت شدن منحنی چندجمله‌ای درجه 8 برروی داده‌ها}{arg3}{0.5}
فرمول منحنی فیت شده در \ref{mon} آمده است.
\begin{equation}
	-1.22 \cdot 10^{-9} x^{8} + 2.03 \cdot 10^{-7} x^{7} - 1.37 \cdot 10^{-5} x^{6} - 0.01 x^{4} + 0.12 x^{3} - 0.94 x^{2} + 3.92 x + 38.75
	\label{mon}
\end{equation}
\قسمت{محاسبه مدول یانگ}
پارامتر‌‌های مربوط به پروتئین به صورت زیر است:
\begin{equation}
	\begin{gathered}
		m = 8564.844 amu\\
		m = 1.4222\cdot10^{-23} kg
		A = 1000 \mathring{A}^2\\
		t = 40 ps\\
		T = 20 ps\\
		F = 10 pN\\
	\end{gathered}
\end{equation}
ابتدا با استفاده از این فرمول به محاسبه ضریب میرایی پروتئین پرداخته می‌شود.
\begin{equation}
	\zeta = \left[1+\left(\frac{2\pi}{ln(y_1/y_2)}\right)^2\right]^{-0.5}
\end{equation}
\begin{equation}
	\omega_d = \frac{2\pi}{T}
\end{equation}
\begin{equation}
	\omega_n = \frac{\omega_d}{\sqrt{1-\zeta^2}}
\end{equation}
\begin{equation}
	2\zeta\omega_n = \frac{C}{m}
\end{equation}
که در آن پارامتر‌‌های به این صورت محاسبه میشوند.
\begin{equation}
	\begin{gathered}
		y_1 = y_{m1} - y_\infty \\
		y_2 = y_{m2} - y_\infty \\
		y_\infty = \frac{1}{N}\sum_{i = 20}^{41}y_i
	\end{gathered}
\end{equation}
برای محاسبه مجانب از میانگین تعداد بیست داده دوم استفاده شده است.
\begin{equation*}
	\begin{gathered}
		y_\infty = 45.1719\cdot10^{10}\\
		y_1 = 45.7822\cdot10^{10}\\
		y_2 = 45.5863\cdot10^{10}
	\end{gathered}
\end{equation*}
ضریب میرایی محاسبه شده 
\begin{equation*}
	\zeta = 0.06149
\end{equation*}
\begin{equation*}
	\omega_d = 31.41\cdot10^{10}
\end{equation*}
\begin{equation*}
	\begin{gathered}
		\omega_n = 31.47\cdot10^{10}\\
		k = 1.40 N/m\\
		y_0 = 38.7525 \mathring{A}\\
		\Delta y = y_\infty - y_0 = 6.4194 \mathring{A}\\
		\eps = \frac{\Delta y}{y_0} = 0.1656\\
		\sigma = \frac{F}{A} = 0.01\cdot10^8 N/m^2\\
		E  = \frac{\sigma}{\eps} = 6.038 MPa
	\end{gathered}
\end{equation*}
 \فصل{ساختار فایل‌‌های ورودی}
\قسمت{فایل‌‌های \لر{PDB}}
این فایل‌ها شامل اطلاعاتی مانند نام‌ترکیب، اورگانیسم، و بافتی که نمونه مربوطه از آن به دست آمده است می‌شود. دو بخش اصلی این فایل بخش \لر{ATM} و \لر{HETATM} است که مربوط به مختصات اتم‌‌های پروتئین، آب، یون‌ها و هرگونه اتم \لر{Heterogeneous} است که در کریستال یافت شده است.
\\
\figs{11}{ساختار فایل \لر{PDB}}{arg3}{0.3}
ساختار این فایل به این صورت است که به‌ترتیب از چپ به راست نوع داده، \لر{ID} اتم،‌ نام اتم،‌ نام \رزیجوی ، \لر{ID} \رزیجوی پارامتر‌‌های مختصات \لر{Occupancy}، فاکتور دمایی،‌نام قسمت و شماره خط را شامل می‌شود.
\قسمت{فایل‌‌های \لر{PSF}}
این فایل‌ها حاوی تمامی اطلاعاتی است که برای اعمال یک میدان نیرو بر روی یک مولکول مورد نیاز خواهد بود. این فایل‌ها دارای شش قسمت مهم هستند. این قسمت‌ها شامل: اتم،‌ پیوند، زوایا، دایهدرال‌ها، \لر{impropers} و \لر{cross term}‌هاست.


