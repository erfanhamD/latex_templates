
\فصل{بررسی مسئله}
\صفحه‌جدید
\قسمت{تعریف مسئله}
\زیرقسمت{هندسه مسئله}
با توجه به هدف مسئله که بررسی جریان مارپیچ در فضای بین دو سیلندر هم‌مرکز است، هندسه مسئله نیز به صورتی که در شکل\رجوع{شکل:هندسه۱}
تعریف شده است بررسی می‌شود

\شروع{شکل}[ht]
\وسط‌چین
\includegraphics[scale=0.3]{"/home/erfan/Videos/advanced_fluid/Project/figures/geom1.png"}
\شرح{هندسه مسئله}
\برچسب{شکل:هندسه۱}
\پایان{شکل}
در این مسئله دو پارامتر $\beta$  و $\alpha$ تعریف می‌شود که $\beta$ محل بیشینه سرعت محوری و $\alpha$ نسبت‌اندازه شعاع خارجی به داخلی است.
\begin{equation*}
	\alpha = \frac{R_i}{R_o}
\end{equation*}
ابعاد و‌اندازه‌های درنظرگرفته شده و موردکاربرد در حل مسئله نیز در شکل
\ref{fig:geom2}
آورده شده است.
\begin{figure}[ht]
	\centering
	\includegraphics[scale=0.3]{../figures/geom2}
	\caption{ابعاد و‌اندازه‌ها}
	\label{fig:geom2}
\end{figure}
\زیرقسمت{\hlc{مدل \لر{UCM}}}
هدف از این نوآوری برای حل مسئله بررسی نتایج تفاوت پارامتر‌های غیرخطی افزوده شده در مدل
\لر{PTT}
نسبت به مدل‌های ساده تری مانند مدل
\لر{UCM}
بوده است.
در مدل
\لر{UCM}
تانسور تنش به صورت غیرصریح به صورت معادله \رجوع{ucm} بیان می‌شود که بالانویس 
\لر{p} اشاره‌گر به این است که تنش‌ها و ویسکوزیته مربوط به بخش پلیمری سیال است و نه بخش محلول آن.
روش  حل تحلیلی به کار رفته در مقاله اصلی به صورت استخراج معادلات غیر خطی با استفاده از فرضیات و سپس حل دستگاه معادلات به‌ دست آمده به روش‌های عددیست.
\cite{old}
\begin{equation}
	\label{ucm}
	\mathbf{T}^{p}+\lambda\left(\frac{\partial \mathbf{T}^{p}}{\partial t}+\overset{\bigtriangledown}{\mathbf{T}^{p}}\right)=2 \eta_{p} \mathbf{D}	
\end{equation}

در این رابطه علامت 
$
\bigtriangledown
$
تحت عنوان مشتق
\lr{upper-convective}
طبق رابطه زیر تعریف می‌شود.
\begin{equation}
		\overset{\bigtriangledown}{\mathbf{T}} = (\mathbf{v}.\nabla)\mathbf{T}-\mathbf{T}(\nabla \mathbf{v}) - (\nabla \mathbf{v})^T\mathbf{T}
\end{equation}
\زیرقسمت{فرضیات}
فرضیات درنظرگرفته شده به منظور حل این مسئله عبارت است از
\شروع{فقرات}
\فقره جریان پایا
\فقره جریان آرام
\فقره خطوط جریان موازی
\فقره جریان توسعه یافته
\فقره تقارن محوری
\فقره همدما بودن جریان
\پایان{فقرات}
\زیرقسمت{معادلات حاکم بر مسئله}

معادلات حاکم بر این مسئله شامل معادله مومنتوم،‌ معادله پیوستگی و معادله مشخصه است، که در دستگاه مختصات استوانه‌ای تعریف میگردند.
\begin{equation}
	\frac{1}{r} \frac{\partial\left(r u_{r}\right)}{\partial r}+\frac{1}{r} \frac{\partial\left(u_{\theta}\right)}{\partial \theta}+\frac{\partial u_{z}}{\partial z}=0
\end{equation}
معادله مومنتوم مورد استفاده، معادله کشی به صورت زیر است.
\begin{equation}
	\rho\frac{Du}{Dt} = -\nabla p + \nabla . \tau
\end{equation}
مومنتوم در جهت $r$
\begin{equation}
	\begin{split}
		&\frac{\partial u_r}{\partial t} + u_r \frac{\partial u_r}{\partial r} + \frac{\nu}{r} \frac{\partial u_r}{\partial\theta} + u \frac{\partial u_r}{\partial z} - \frac{\nu^2}{r}\\
		&= -\frac{1}{\rho} \frac{\partial P}{\partial r} + \frac{1}{r\rho}\frac{\partial\left(r\tau_{rr}\right)}{\partial r} + \frac{1}{r\rho} \frac{\partial\tau_{\theta r}}{\partial\theta} + \frac{1}{\rho} \frac{\partial\tau_{zr}}{\partial z} - \frac{\tau_{\theta\theta}}{r\rho} 
	\end{split}
\end{equation}
\\
مومنتوم در جهت $\theta$
\begin{equation}
	\begin{split}
		&\frac{\partial \nu}{\partial t} + \nu \frac{\partial \nu}{\partial r} + \frac{\nu}{r} \frac{\partial \nu}{\partial\theta} + u \frac{\partial \nu}{\partial z} + \frac{\nu \nu}{r}\\
		&= -\frac{1}{r\rho} \frac{\partial P}{\partial\theta} + \frac{1}{r\rho}\frac{\partial\tau_{\theta \theta}}{\partial\theta} + \frac{1}{r^2 \rho} \frac{\partial\left(r^2 \tau_{r\theta}\right)}{\partial r} + \frac{1}{\rho} \frac{\partial\tau_{z\theta}}{\partial z} 
	\end{split}
\end{equation}
\\
مومنتوم در جهت $z$
\begin{equation}
	\begin{split}
\frac{\partial u}{\partial t} + \nu \frac{\partial u}{\partial r} + \frac{\nu}{r} \frac{\partial u}{\partial\theta} + u \frac{\partial u}{\partial z}
= -\frac{1}{\rho} \frac{\partial P}{\partial z} + \frac{1}{\rho} \frac{\partial\tau_{zz}}{\partial z} + \frac{1}{r\rho}\frac{\partial\tau_{\theta z}}{\partial\theta} + \frac{1}{r\rho}\frac{\partial\left(r\tau_{rz}\right)}{\partial r}
	\end{split}
\end{equation}
با توجه به فرضیات روابط مومنتوم به صورت زیر ساده می‌شوند:
\\
مومنتوم در جهت $r$
\begin{equation}
	\label{rmom}
	\begin{split}
		- \frac{\nu^2}{r}
		= -\frac{1}{\rho} \frac{\partial P}{\partial r} + \frac{1}{r\rho}\frac{\partial\left(r\tau_{rr}\right)}{\partial r} - \frac{\tau_{\theta\theta}}{r\rho} 
	\end{split}
\end{equation}
\\
مومنتوم در جهت $\theta$
\begin{equation}
	\label{tetmom}
	\begin{split}
		0
		= \frac{\partial\left(r^2 \tau_{r\theta}\right)}{\partial r}
	\end{split}
\end{equation}
\\
مومنتوم در جهت $z$
\begin{equation}
	\label{zmom}
	\frac{1}{r} \frac{\mathrm{d}\left(r \tau_{r z}\right)}{\mathrm{d} r}=\frac{\partial p}{\partial z}
\end{equation}
\begin{multicols}{2}
معادلات مشخصه نیز به صورت زیر ساده می‌شوند.
\begin{eqnarray}
\tau_{rr} &= 0 \label{112}\\
\tau_{zz} &= 2\lambda\eta(\frac{du}{dr})^2\\
\tau_{\theta \theta} &= 2\lambda\eta\left(r\frac{d\left(\nu/r\right)}{dr}\right)^2
\end{eqnarray}

\begin{eqnarray}
	\tau_{r \theta} &= r\eta \frac{d\left(\nu /r\right)}{dr}\label{142}\\
	\tau_{rz} &= \eta \frac{du}{dr} \label{15-2}\\
	\tau_{\theta z} &= 2\lambda\eta r \frac{du}{dr}\frac{d\left(\nu /r\right)}{dr}
\end{eqnarray}
\end{multicols}
با وارد کردن پارامتر گشتاور بر واحد طول $M$ معادله زیر حاصل می‌شود.
\begin{equation}
\label{taurtheta}
\tau_{r \theta} = \frac{M}{2\pi r^2}
\end{equation}
حال اگر رابطه
\ref{taurtheta}
در رابطه 
\ref{142}
قرار داده شود برای سرعت شعاعی معادله دیفرانسیل زیر به دست خواهد‌ آمد.
\begin{equation}
	\label{15}
	r\frac{d}{dr}\left(\frac{\nu}{r}\right) = \frac{M}{2\pi\eta r^2}
\end{equation}
%با قرار دادن رابطه به دست آمده در رابطه 
%\ref{1-8}
%المان تنش $\tau_{\theta \theta}$ بدین صورت به دست می‌آید
%\begin{equation}
%	\label{1-15}
%	\tau_{\theta \theta} = \frac{\lambda M^2}{2\pi ^2 \eta r^4}
%\end{equation}
%حال اگر رابطه 
%\ref{1-15}
%در مولفه شعاعی مومنتوم 
%\ref{1-10}
%قرار داده شود. رابطه توزیع شعاعی فشار به صورت زیر به دست می‌آید
%\begin{equation}
%	\label{16}
%	r\frac{\partial p}{\partial r} = \rho \nu ^2 - \frac{\lambda M^2}{2\pi ^2\eta r^4}
%\end{equation}
%همانطور که مشخص است این رابطه وابسته به مولفه شعاعی سرعت نیز هست. بنابراین با به دست آوردن مولفه شعاعی سرعت توزیع شعاعی فشار نیز به دست می‌آید.
اگر از معادله
\ref{zmom}
انتگرال گرفته شود رابطه زیر حاصل می‌شود.
\begin{equation}
	\label{17}
	\tau_{r z} = \frac{\partial p }{\partial z}\frac{r}{2} + \frac{c_2}{r}
\end{equation}
و از مساوی قرار دادن رابطه 
\ref{17}
با رابطه  
\ref{15-2}
رابطه
\ref{18}
به دست می‌آید.
\begin{equation}
	\label{18}
	\eta \frac{du}{dr} = \frac{\partial p }{\partial z}\frac{r}{2} + \frac{c_2}{r}
\end{equation}
حال 
با فرض اینکه تنش برشی $\tau_{r z}$ در $r = \beta R$ برابر صفر باشد. ثابت انتگرال‌گیری $c_2 $ به دست می‌آید و با جایگذاری در رابطه  
\ref{17} 
نتیجه خواهدشد
\begin{equation}
	\tau_{r z} = \frac{1}{2}\frac{\partial p \beta R}{\partial z}\left(\frac{\beta R}{r}-\frac{r}{\beta R}\right)
\end{equation}
\\
در ادامه به جهت خلاصه نویسی و تطابق با متن مقاله به جای 
$\frac{\partial p}{\partial z}$
از
$p_z$
استفاده می‌شود.
در نهایت رابطه گرادیان مولفه‌های سرعت در جهت $z$ و $\theta$ به صورت رابطه‌های
\ref{19}
و
\ref{20} 
به دست می‌آیند.

\begin{equation}
	\label{19}
		\eta \frac{du}{dr} = \left[-\frac{p_{z} \beta R}{2}\left(\frac{\beta R}{r}-\frac{r}{\beta R}\right)\right]
\end{equation}
\begin{equation}
	\label{20}
	r\frac{d}{dr}\left(\frac{\nu}{r}\right) = \frac{M}{2\pi\eta r^2}
\end{equation}
%\زیرقسمت{بی‌ بعد سازی متغیرها}
%اگر سرعت میانگین سطح مقطع را $U$، و سرعت مشخصه شعاعی را به صورت 
%\begin{equation}
%	U_{T}=\frac{M}{\pi \eta \zeta}
%\end{equation}
%در نظر بگیریم که در آن پارامتر $\zeta$ به عنوان طول مشخصه به صورت زیر تعریف شده باشد:
% \begin{equation}
% 	\zeta = R - \alpha R = R(1-\alpha)
% \end{equation}
%می‌توان اعداد بدون بعد زیر را تعریف کرد
%\begin{equation}
%	De = \frac{\lambda U}{\zeta}
%\end{equation}
%\begin{equation}
%	De_T = \frac{\lambda U_T}{\zeta}
%\end{equation}
%\begin{equation}
%	\bar{p}_z = \frac{p_z \zeta ^2}{\eta U}
%\end{equation}
%\begin{equation}
%	\bar{u}  = \frac{u}{U}
%\end{equation}
%\begin{equation}
%	\bar{\nu} = \frac{\nu}{U}
%\end{equation}
%\begin{equation}
%	\bar{r} = \frac{r}{\zeta}
%\end{equation}
حال معادله گرادیان سرعت در راستای محوری و مماسی را به صورت بی بعد بازنویسی میکنیم.
\begin{equation}
	\label{dudr}
	\frac{\mathrm{d} \bar{u}}{\mathrm{~d} \bar{r}}=\left[-\frac{\bar{p}_{z} \beta \bar{R}}{2}\left(\frac{\beta \bar{R}}{\bar{r}}-\frac{\bar{r}}{\beta \bar{R}}\right)\right]
\end{equation}
\begin{equation}
	\label{dnudr}
	\frac{\mathrm{d}(\bar{\nu} / \bar{r})}{d \bar{r}}=\frac{U_{T}}{U} \frac{1}{2 \bar{r}^{3}}
\end{equation}
حال از روابط 
\ref{dudr}
و 
\ref{dnudr}
انتگرال گرفته می‌شود و پروفیل شعاعی و مماسی سرعت به دست می‌آید.
\begin{equation}
	\label{uC}
	\bar{u}(\bar{r}) = \bar{u_a}(\bar{r}) + \bar{c_b}
\end{equation}
\begin{equation}
	\label{nuC}
	\bar{\nu}(\bar{r}) = \bar{r}\bar{\nu_a}(\bar{r}) + \bar{c_c}\bar{r}
\end{equation}
که در این روابط پارامتر $\bar{u_a}(\bar{r})$ بدین صورت به دست آمده است.
\begin{equation}
	\label{uwithc}
	\bar{u_a}(\bar{r}) = \frac{r^{2} \bar{p}_{z}}{4}-\frac{\beta^{2} \bar{R}^{2} \bar{p}_{z} \ln (\bar{r})}{2}
\end{equation}
\begin{equation}
	\label{nuwithc}
	\bar{\nu}_a(\bar{r})=-\frac{U_{T}}{4 r^{2} U}
\end{equation}
به منظور به دست آوردن ثابت‌های انتگرال‌گیری باید شرایط مرزی در معادلات اصلی جایگذاری شوند.
\زیرقسمت{شرایط مرزی}
شرط مرزی لغزش سرعت محوری بر روی مرز سیلندر داخلی را می‌توان به این صورت تعریف کرد
\begin{equation}
	u(\alpha R)=k_{i}\left|\tau_{r z}\right|^{m_{i}-1} \tau_{r z}
\end{equation}
و شرط مرزی لغزش روی مرز سیلندر خارجی به صورت زیر تعریفمی‌گردد.
\begin{equation}
	u(R)=-k_{o}\left|\tau_{r z}\right|^{m_{o}-1} \tau_{r z}
\end{equation}
که در این روابط ثابت‌های $k$ و $m$ به ترتیب بیانگر ثابت قانون لغزش و توان قانون لغزش هستند و در هرکدام از شرایط مرزی روی سیلندر داخلی و خارجی می‌توانند مقادیر مختلفی داشته باشند. در این روابط پاورقی $i$ و $o$ به ترتیب اشاره‌گر به سیلندر داخلی و خارجی هستند. این شرایط مرزی را می‌توان به صورت بدون بعد نوشت.
\begin{equation}
	\label{innerbc}
		\bar{u}(\alpha \bar{R})=\bar{k}_{i}\left[-\bar{p}_{. z} \frac{\beta \bar{R}}{2}\left(\frac{\beta}{\alpha}-\frac{\alpha}{\beta}\right)\right]^{m_{i}}
\end{equation}
\begin{equation}
	\label{outerbc}
		\bar{u}(\bar{R})=\bar{k}_{o}\left[\bar{p}_{z} \frac{\beta \bar{R}}{2}\left(\beta-\frac{1}{\beta}\right)\right]^{m_{o}}
\end{equation}
%که در این روابط 
%\begin{equation*}
%	\bar{k}_{i}=\left(\frac{k_{i} \eta}{\zeta}\right)\left(\frac{|U|\eta} {\zeta}\right)^{m_{i}-1}
%\end{equation*}
%\begin{equation*}
%	 \bar{k}_{o}=\left(\frac{k_{o} \eta}{\zeta}\right)\left(\frac{|U| \eta}{\zeta}\right)^{m_{o}-1}
%\end{equation*}
با قرار دادن شرط مرزی
\ref{innerbc}
در معادله 
\ref{uwithc}
مقدار ضریب $c_b$ به دست آمده و رابطه بی بعد سرعت محوری به صورت زیر به دست می‌آید.
\begin{equation}
	\label{27}
	\bar{u}(\bar{r})=\bar{u}_{a}(\bar{r})-\bar{u}_{a}(\alpha \bar{R})+\bar{k}_{i}\left[-\bar{p}_{z} \frac{\beta \bar{R}}{2}\left(\frac{\beta}{\alpha}-\frac{\alpha}{\beta}\right)\right]^{m_{i}}
\end{equation}
حال برای محاسبه پارامتر $\beta$ شرط مرزی
\ref{outerbc}
در معادله قرار داده می‌شود.
\begin{equation}
	\label{f1}
	\begin{aligned}
		f_{1}\left(\frac{U_{T}}{U}, \bar{p}_{z}, \beta\right) =  &  \bar{u}_{a}(\bar{R})-\bar{u}_{a}(\alpha \bar{R})+\bar{k}_{i}\left[-\bar{p}_{z} \frac{\beta \bar{R}}{2}\left(\frac{\beta}{\alpha}-\frac{\alpha}{\beta}\right)\right]^{m_{i}} \\
		&-\bar{k}_{o}\left[\bar{p}_{z} \frac{\beta \bar{R}}{2}\left(\beta-\frac{1}{\beta}\right)\right]^{m_{o}}=0
	\end{aligned}
\end{equation}
حال اگر رابطه پروفیل سرعت محوری در رابطه
\ref{f1}
جایگذاری شود برای  $m_i=m_o= m$‌های مختلف روابط زیر به دست می‌آید.
\\
برای $ m = 1$
\begin{equation}
	\begin{aligned}
		\label{f1_1}
		&f_{1}\left(\frac{U_{T}}{U}, \bar{p}_{z}, \beta\right) =\\
		&\frac{R p_{z} \left(\alpha \left(- R \left(\alpha^{2} - 2 \beta^{2} \log{\left(R \alpha \right)}\right) - R \left(2 \beta^{2} \log{\left(R \right)} - 1\right) + 2 k_{o} \left(\beta^{2} - 1\right)\right) + 2 k_{i} \left(\alpha^{2} - \beta^{2}\right)\right)}{4 \alpha}
	\end{aligned}	
\end{equation}
\hlc{اگر رابطه بالا به صورت صریح بر حسب پارامتر $\beta$ نوشته شود}
\begin{equation}
	\beta = \left[\frac{\alpha \left(- R \alpha^{2} + R + 2 \alpha k_{i} - 2 k_{o}\right)}{2 \left(R \alpha \log{\left(R \right)} - R \alpha \log{\left(R \alpha \right)} - \alpha k_{o} + k_{i}\right)}\right]^\frac{1}{2}
\end{equation}

برای $m = 2$ و $m=3 $ نیز پارامتر $\beta$ به صورت صریح محاسبه می‌شود.
حال برای به دست آوردن پارامتر‌های افت فشار و پروفیل سرعت محوری و مماسی از یک رابطه دیگر برای محاسبه سرعت محوری میانگین استفاده می‌شود.
\begin{equation}
	U=\frac{1}{\pi R^{2}\left(1-\alpha^{2}\right)} \int_{\alpha R}^{R} u(r) 2 \pi r d r
\end{equation}
حال اگر این رابطه به صورت بدون بعد نوشته شود:
\begin{equation}
	\label{aveU}
	\frac{2}{\bar{R}^{2}\left(1-\alpha^{2}\right)} \int_{\alpha \bar{R}}^{\bar{R}} \bar{u}(\bar{r}) \bar{r} d \vec{r}-1=0
\end{equation}
\hlc{اگر با استفاده از رابطه 
\ref{uwithc}
از رابطه 
\ref{aveU}
انتگرال گرفته شود رابطه زیر به دست می‌آید:}
%\begin{landscape}
%		\begin{sidewaysfigure}
%	\tiny
	\begin{equation}
		\begin{aligned}
		&f_{2}\left(\frac{U_{T}}{U}, \bar{p}_{z}, \beta\right) =\\
		&\frac{1}{\alpha^2-1}(- \frac{R^{2} \alpha^{4} p_{z}}{8} + \frac{R^{2} \alpha^{2} \beta^{2} p_{z}}{4} + \frac{R^{2} \alpha^{2} p_{z}}{4} + \frac{R^{2} \beta^{2} p_{z} \log{\left(R \right)}}{2} - \frac{R^{2} \beta^{2} p_{z} \log{\left(R \alpha \right)}}{2} - \frac{R^{2} \beta^{2} p_{z}}{4} - \frac{R^{2} p_{z}}{8} +\\ &\alpha^{2} k_{i} \left(\frac{R p_{z} \left(\alpha^{2} - \beta^{2}\right)}{2 \alpha}\right)^{m_{i}} - \alpha^{2} - k_{i} \left(\frac{R p_{z} \left(\alpha^{2} - \beta^{2}\right)}{2 \alpha}\right)^{m_{i}} + 1)
	\end{aligned}
	\end{equation}
%	\end{sidewaysfigure}
حال به منظور به دست آوردن پارامتر ثابت $C_c$ در رابطه
\ref{nuC}
شرط مرزی لغزش برای سیلندر داخلی
\begin{equation}
	\nu(\alpha R) = \omega \alpha R + k_{\nu i}||\tau_{r \theta}||^{m_{\nu i}-1}\tau_{r \theta}
\end{equation}
و همچنین برای سیلندر خارجی
\begin{equation}
	\nu(R) = k_{\nu o}||\tau_{r \theta}||^{m_{\nu o}-1}\tau_{r \theta}
\end{equation}
در پروفیل سرعت شعاعی اعمال می‌شوند. بدین منظور لازم است که ابتدا این شرایط مرزی به صورت بی بعد نوشته شوند.
\begin{equation}
	\label{nu_alphaR}
	\bar{v}(\alpha \bar{R})=\bar{\omega} \alpha \bar{R}+\bar{k}_{v i} \frac{\left|\frac{U_{T}}{U}\right|^{m_{\nu j}-1}\left(\frac{U_{T}}{U}\right)}{(\alpha \bar{R})^{2 m_{t i}}}
\end{equation}
\begin{equation}
	\label{nu_R}
	\bar{v}(\bar{R})=-\bar{k}_{\nu o} \frac{\left|\frac{U_{T}}{U}\right|^{m_{\nu o}-1}\left(\frac{U_{T}}{U}\right)}{(\bar{R})^{2 m_{k o}}}
\end{equation}
%که در این روابه پارامتر‌ها بدین صورت بدون بعد شده‌اند.
%\begin{equation*}
%	\bar{k}_{\nu i}=\frac{k_{\nu i}}{|U|}\left(\frac{\eta|U|}{2 \zeta}\right)^{m_{\nu i}}
%\end{equation*}
%\begin{equation*}
%	\bar{k}_{\nu o}=\frac{k_{\nu o}}{|U|}\left(\frac{\eta|U|}{2 \zeta}\right)^{m_{\nu o}}
%\end{equation*}
%\begin{equation*}
%	\bar{\omega}=\frac{\omega \zeta}{U}
%\end{equation*}
مقادیر ثابت‌های لغزش در شرط مرزی‌ها برای سرعت‌های محوری و مماسی می‌تواند با یکدیگر متفاوت باشند.
\\
حال می‌توان پروفیل سرعت مماسی را به دست آورد.
\begin{equation}
	\label{nu_profile}
	\bar{v}(\bar{r})=\bar{r} \bar{v}_{r}(\bar{r})-\bar{r} \bar{v}_{r}(\bar{R})-\bar{k}_{\nu_{o}} \frac{\left|\frac{U_{T}}{U}\right|^{m_{\nu o}-1}\left(\frac{U_{T}}{U}\right)}{(\bar{R})^{2 m_{\nu_{o}}}} \frac{\bar{r}}{\bar{R}}
\end{equation}
\hlc{با قرار دادن رابطه
\ref{nu_alphaR}
در رابطه 
\ref{nu_profile}
سومین معادله اصلی حاصل می‌شود:}
\begin{equation}
	\begin{aligned}
		f_{3}\left(\frac{U_{T}}{U}, \bar{p}_{, z}, \beta\right) \equiv & \bar{v}_{r}(\alpha \bar{R}) \alpha \bar{R}-\bar{v}_{r}(\bar{R}) \alpha \bar{R}-\bar{k}_{\nu o} \frac{\left|\frac{U_{T}}{U}\right|^{m_{\nu o}-1}\left(\frac{U_{T}}{U}\right)}{(\bar{R})^{2 m_{\nu o}}} \alpha \\
		&-\left[\bar{\omega} \alpha \bar{R}+\bar{k}_{\nu i} \frac{\left|\frac{U_{T}}{U}\right|^{m_{v i}-1}\left(\frac{U_{T}}{U}\right)}{(\alpha \bar{R})^{2 m_{v i}}}\right]=0
	\end{aligned}
\end{equation}
حال برای به دست آوردن پروفیل‌های سرعت محوری و مماسی و پارامتر‌های افت‌ فشار و $\beta$ میبایست این سه معادله غیرخطی را به صورت همزمان حل کرد.
\begin{equation}
	\left\{\begin{array}{l}
		f_{1}\left(\frac{U_{T}}{U}, \bar{p}_{z}, \beta\right)=0 \\
		f_{2}\left(\frac{U_{T}}{U}, \bar{p}_{z}, \beta\right)=0 \\
		f_{3}\left(\frac{U_{T}}{U}, \bar{p}_{z}, \beta\right)=0
	\end{array}\right.
\end{equation}
\newpage
\قسمت{\hlc{نتایج}}
در معادلات به دست‌آمده برای مدل
\lr{UCM}
پارامتر 
$\varepsilon$
 صفر شده و نتیجتا پارامتر 
 $\varepsilon De^2$
ظاهر نمی‌شود. بنابراین مقایسه پارامتر‌ها به صورت نموداری در بعضی موارد ممکن نبوده چرا که داده‌های به دست آمده از مدل 
\lr{UCM}
فقط یک نقطه از روی نمودار را نمایش میداد.
نتایج به دست آمده نشان دهنده این است که مدل
\lr{UCM}
در حالت پایا رفتار یک سیال نیوتونی را از خود نشان میدهد از این نظر شاید بتوان گفت که این مدل برای مدل سازی سیالات 
\lr{Shear Thinning}
و سیالات
\lr{Thixotropic}
مناسب باشد.
\شروع{شکل}[h!]
\وسط‌چین
\includegraphics[scale=0.6]{"/home/erfan/Videos/advanced_fluid/Project/figures/fig5.eps"}
\شرح{تنش برشی $\tau_{r z}$بر حسب شعاع نسبی برای پارامتر‌های $\alpha = 0.5$ و $m=1$ و $ k_i=k_o=0$ برای مدل خطی و سهمی $PTT$}
\برچسب{شکل:8}
\پایان{شکل}
\begin{figure}[h!]
	\centering
	\begin{minipage}{0.45\textwidth}
		\includegraphics[scale=0.5]{"/home/erfan/Videos/advanced_fluid/Project/figures/fig8.eps"}
		\caption{\small
			پروفیل سرعت محوری نسبت به شعاع با پارامتر‌های $\alpha = 0.5$ و $m=1$ و $ k_i=k_o=k$ برای مدل خطی $PTT$}
	\end{minipage}\hfill
	\begin{minipage}{0.45\textwidth}
		\includegraphics[scale=0.5]{"/home/erfan/Videos/advanced_fluid/Project/figures/fig9a_graph.eps"}
		\شرح{\small
			پروفیل سرعت محوری نسبت به شعاع با پارامتر‌های $\alpha = 0.5$ و $m=1$ و $ k_o=0$ برای مدل خطی $PTT$}
	\end{minipage}
\end{figure}
\\
\begin{figure}[ht!]
	\centering
	\includegraphics[scale=0.6]{"/home/erfan/Videos/advanced_fluid/Project/figures/fig12b_graph.eps"}
	\شرح{
		\small
		پروفیل سرعت بی‌بعد مماسی نسبت به شعاع با پارامتر‌های $\alpha = 0.5$ و $m=1$ و $ k_i=k_o=k$ برای مدل خطی و سهمی $PTT$}
\end{figure}
\begin{figure}

\end{figure}
%\begin{equation}
%\nu (r) = - \frac{0.125 R^{4} U_{T} r}{U} + \frac{0.125 U_{T} r^{5}}{U} - \frac{R^{- 2 m_{i}} U_{T} k_{o} r \left(\frac{U_{T}}{U}\right)^{m_{o} - 1}}{R U}
%\end{equation}
%\begin{equation}
%	f_3 = \frac{0.125 R^{5} U_{T} \alpha^{5}}{U} - \frac{0.125 R^{5} U_{T} \alpha}{U} - R \alpha \omega - \frac{U_{T} k_{i} \left(R \alpha\right)^{- 2 m_{i}} \left(\frac{U_{T}}{U}\right)^{m_{i} - 1}}{U} - \frac{R^{- 2 m_{i}} U_{T} \alpha k_{o} \left(\frac{U_{T}}{U}\right)^{m_{o} - 1}}{U}
%\end{equation}
%\begin{equation}
%	\frac{0.125 R^{5} U_{T} \alpha^{5}}{U} - \frac{0.125 R^{5} U_{T} \alpha}{U} - 0.1 - \frac{U_{T} k_{i} \left(R \alpha\right)^{- 2 m_{i}} \left(\frac{U_{T}}{U}\right)^{m_{i} - 1}}{U} - \frac{R^{- 2 m_{i}} U_{T} \alpha k_{o} \left(\frac{U_{T}}{U}\right)^{m_{o} - 1}}{U}
%\end{equation